\documentclass[brazil, a4paper,12pt]{article}
\usepackage[brazil]{babel}
\usepackage{graphicx}
\usepackage{geometry}
\usepackage[utf8]{inputenc}
\usepackage[T1]{fontenc}
\usepackage{listings}
\usepackage{indentfirst}
\geometry{a4paper,left=3cm,right=3cm,top=2.5cm,bottom=2.93cm}

\lstset{numbers=left,
stepnumber=1,
firstnumber=1,
numberstyle=\footnotesize,
extendedchars=true,
breaklines=true,
frame=tb,
tabsize=2,
basicstyle=\footnotesize,
stringstyle=\ttfamily,
showstringspaces=false
}

\renewcommand{\lstlistingname}{Código}

\begin{document}
\begin{titlepage}

  \vfill

  \begin{center}
    \begin{large}
      Universidade Federal de Ouro Preto
    \end{large}
  \end{center}

  \begin{center}
    \begin{large}
      Instituto de Ciências Exatas e Aplicadas
    \end{large}
  \end{center}

  \begin{center}
    \begin{large}
      Departamento de Computação e Sistemas
    \end{large}
  \end{center}

  \vfill

  \begin{center}
    \begin{Large}
      \textbf{Redes de Computadores 1\\[0.4cm] 
        Trabalho Prático 1}               
    \end{Large}
  \end{center}


  \vfill

  \begin{center}
    \begin{large}
      Guilherme Marx Ferreira Tavares
    \end{large}
  \end{center}

  \begin{center}
    \begin{large}
      Professor - Theo Silva Lins
    \end{large}
  \end{center}

  \vfill

  \begin{center}
    \begin{large}
      João Monlevade \\
      \today \\
    \end{large}
  \end{center}

\end{titlepage}

%--------------------------------------------------------

\tableofcontents 
\newpage
\section{Introdução}
Desde o início da computação, foi necessária criar uma comunicação entre máquinas diferentes. Para padronizar essa comunicação, foram criados diversos protocolos. Hoje, temos a rede mundial de computadores, onde as máquinas estão todas conectadas por esses protocolos.
\newpage
\section{Objetivos}
Esse trabalho tem como objetivo aplicar o conceito de socket em rede de Computadores. Será implementada uma comunicação entre dois sistemas diferentes.

O cliente dessa comunicação irá receber uma sequência de 10 números inteiros e enviá-los ao Servidor. O Servidor deverá ordenar esses números inteiros e retornar ao Cliente.
\newpage
\section{Implementação}
Primeiramente, é iniciado um servidor na porta 55555(vide códigos no apêndice). Esse servidor ficará esperando a conexão do Cliente.

No Cliente, primeiramente é criado um Scanner para receber as entradas do teclado que o usuário digitará. Em seguida, inicia-se uma conexão com o Servidor e cria-se os canais de entrada e saída de dados.

Em seguida o cliente envia os dados recebidos pelo Scanner pro Servidor.

Novamente no servidor, esses dados recebidos como string primeiramente devem ser "cortados" em strings menores a fim de transformar esses números recebidos como chars em inteiros. Uma vez passados para inteiros, o comando Arrays.sort(vetor) é utilizado pra ordenar esse vetor. O vetor ordenado é passado, então, para uma string e retornado pro Cliente. Com isso, o Servidor encerra seu ciclo de operação.

No Cliente, é recebido esse valor e exibido na tela e, em seguida, é fechada a conexão.
\newpage
\section{Apêndice}
\lstinputlisting[language=java, label=cod:Cliente, caption={mainCliente.java}]{mainCliente.java}
\lstinputlisting[language=java, label=cod:Servidor, caption={mainServidor.java}]{mainServidor.java}


\end{document}
